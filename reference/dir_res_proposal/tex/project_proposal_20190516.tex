\documentclass{article}
\usepackage[utf8]{inputenc}
\usepackage[margin=1in]{geometry}

\renewcommand{\familydefault}{\sfdefault}
\renewcommand\textbullet{\ensuremath{\bullet}}

\usepackage{lmodern}
\usepackage{float}
\usepackage{titling}
\usepackage{lineno}
\usepackage{tikz}
\usepackage{multicol}
\usepackage{booktabs}
\usepackage{import}
\usepackage{xifthen}
\usepackage{pdfpages}
\usepackage{transparent}

\newcommand{\incfig}[1]{%
    \def\svgwidth{\columnwidth}
    \import{./figs/}{#1.pdf_tex}
}

\setlength{\droptitle}{-5em}

\title{BICB8994 Project Proposal}
\author{Christopher Tastad}

\begin{document}

\maketitle

\par Sensitivity to chemotherapy stands as a major hurdle to effective treatment in women with ovarian cancer. The presentation of a resistant cancer lineage poses a significant threat to patients in the form of notably lower survival outcomes and is ultimately not well understood. To address this, we seek to take a targeted informatics approach to evaluate the nature of ovarian cancer sensitivity and identify routes to selectively regain treatment potency. In this effort, we have collected several data sets from a group of 50 ovarian cancer positive patients, which includes longitudinal observation of their disease progression. From these data sets, we aim to develop a cross-referenced model using deep insight from single cell sequencing among broader whole exome sequencing for comprehensive molecular characterization. More specifically, we will apply a full breadth of analysis across single cell RNAseq, bulk RNAseq, and whole exome data to establish an expansive cross section of the genetic profile of chemotherapy response. Additionally, these findings will be developed in conjunction with the clinical outcomes of our patient population by drawing contrast between individuals who experience resistance in their personal therapy response and those who do not. Further considerations for applying this work include the ability to generate patient-derived xenograph mice to serve as avatars in exploration of drug efficacy. With this analysis we hope to produce a personalized approach to identify and address chemotherapy resistance in an effort to elevate treatment outcomes.

\end{document}
